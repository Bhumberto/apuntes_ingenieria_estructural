\section{Estructuracion}
Según \cite{san1998analisis}, la estructuración de un edificio consiste en “tomar decisiones en conjunto con los otros profesionales que intervienen en la obra acerca de la disposición y características que deben tener los diferentes elementos estructurales, de manera que el edificio tenga un buen comportamiento durante su vida útil; esto es, que tanto las cargas permanentes (peso propio, acabados, etc.) como las eventuales (sobrecarga, sismo, viento, etc.), y se transmitan adecuadamente hasta el suelo de cimentación”.\\
\section{Predimensionamiento}
\subsection{Losas Aligeradas}
Según \cite{blanco} el peralte de las losas aligeradas podrá ser dimensionado considerando los siguientes criterios:
\newpage
\begin{table}[htbp]
  \centering
  \caption{Peso propio y espesores recomendados en aligerados}
  \vspace{0.15cm}
{
\extrarowheight = -0.5ex
\renewcommand{\arraystretch}{1.8}

\begin{tabular}{|>{\centering\arraybackslash}m{2cm}|>{\centering\arraybackslash}m{5cm}| >{\centering\arraybackslash}m{5cm}|}
 %\begin{tabular}{|c|c|c|}
    \hline
    \textbf{h (m)} & 
    \textbf{Peso propio aproximado (kg/m\raisebox{1ex}{\scriptsize{2}})} & 
    \textbf{Luces máximas recomendadas (m)} \\
    \hline
    0.17  & 280   & ln $\leq$ 4 \\
    0.20   & 300   & 4 $\leq$ ln $\leq$ 5.5 \\
    0.25  & 350   & 5.5 $\leq$ ln $\leq$ 6.6 \\
    0.30   & 420   & 6 $\leq$ ln $\leq$ 7.5 \\
    \hline
 \end{tabular}%
}
  \caption*{\small Fuente: \it \cite{blanco}}
  \label{tab:addlabel}%
\end{table}%
\noindent
Se entiende por h al espesor total del aligerado incluyendo los 5cm de losa superior.
\\
El criterio anterior sólo aplica para sobrecargas máximas de 300 a 350 kg/m\raisebox{1ex}{\scriptsize{2}}.
\\
Teniendo en cuenta estos criterios se adopto losas aligeradas armadas en el sentido paralelo a los ejes A,B y C. El espesor entre los ejes 1 y 3 es de 17cm y entre los ejes 4 y 6 es de 20cm.

\subsection{Losas Macizas}
Se adopto una losa maciza en la zona donde existe discontinuidad del diafragma debido a los ductos y la presencia de la escalera. Se adopto un peralte de 20cm con la intención de trasmitir las fuerzas sísmicas del diafragma adecuadamente a los demás elementos.

\subsection{Vigas principales}
Se provee al edificio de vigas peraltadas en las dos direcciones ``X'' e ``Y'', a manera de contar con suficiente rigidez lateral ante un evento sísmico y trabajen de manera conjunta como pórtico y/o pórtico-placa.
\\
Según \cite{blanco}, las vigas se dimensionan generalmente considerando un peralte del orden de 1/10 a 1/12 de la luz libre.
\\
Las vigas dimensionadas con este criterio son diseñadas solo con acero en tracción y no existe problemas de deflexiones grandes.
La norma peruana E 060 indica que el ancho mínimo de vigas que forman parte de elementos sismorresistentes debe ser 25cm.
\\
Las luces libres de máxima longitud son de aproximadamente de 4m por lo que según este criterio solo sería necesario peraltes del orden de 40cm, sin embargo, después de realizar el análisis sísmico se adoptó vigas principales de 30x50cm para cumplir con los requisitos que se mencionan posteriormente.

\subsection{Vigas secundarias}
Al igual que en el caso anterior las dimensiones finales de las vigas secundarias son por requerimiento de rigidez lateral resultando estas 25x40cm.

\subsection{Columnas}

Según \cite{ovi2016} las dimensiones de las columnas se pueden estimar con la expresión:

\begin{equation}
A_{c}=\frac{\lambda\;P_{g} }{\eta \;f'_{c}}
\end{equation}
\myequations{Predimensonamiento de columnas}

\begin{flushleft}
Donde:\\
$\lambda$, $\eta$ = Factores obtenidos de la tabla .\\
$P_{g}$ = Carga gravitacional repartida por m\raisebox{1ex}{\scriptsize{2}}, se puede asumir 1 ton/m\raisebox{1ex}{\scriptsize{2}}\\
$f'_{c}$ = Resistencia a compresión del concreto.\\
\end{flushleft}

% Table generated by Excel2LaTeX from sheet 'Hoja1'
\begin{table}[htbp]
  \centering
  \caption{Predimensionamiento de columnas}
    \begin{tabular}{|c|c|c|}
    \hline
    \rowcolor[rgb]{ .906,  .902,  .902} \textit{\textbf{Tipo de columna:}} & \multicolumn{1}{p{10.665em}|}{\centering\textbf{$\lambda$}} & \multicolumn{1}{p{11.945em}|}{\centering\textbf{$\eta$}} \\
    \hline
    \rowcolor[rgb]{ .906,  .902,  .902} Central & \cellcolor[rgb]{ 1,  1,  1}1.10 & \cellcolor[rgb]{ 1,  1,  1}0.30 \\
    \hline
    \rowcolor[rgb]{ .906,  .902,  .902} Perimetral & \cellcolor[rgb]{ 1,  1,  1}1.25 & \cellcolor[rgb]{ 1,  1,  1}0.25 \\
    \hline
    \rowcolor[rgb]{ .906,  .902,  .902} Esquinera & \cellcolor[rgb]{ 1,  1,  1}1.50 & \cellcolor[rgb]{ 1,  1,  1}0.20 \\
    \hline
    \end{tabular}%
  \label{tab:addlabel}%
\end{table}%

Después de aplicar la ecuación  para la columnas con mayor área tributaria se obtuvo dimensiones mínimas, sin embargo las dimensiones finales de las columnas son por requerimiento de rigidez lateral, resultado estas de 30x65 peraltadas en la dirección paralela al eje X.


\subsection{Muros de corte o placas}

La longitud final de los muros en ambas direcciones se establece después de realizar un análisis sísmico iterativo hasta cumplir con los requisitos de rigidez lateral de la norma E-030.